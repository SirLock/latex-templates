%!TEX TS-program = pdflatex
%!TEX TS-options = -shell-escape
% Author: Phil Steinhorst, p.st@wwu.de
% https://github.com/phist91/latex-templates

\newcommand{\obenlinks}{Name der Vorlesung}		% hier Name der Veranstaltung eintragen
% Author: Phil Steinhorst, p.st@wwu.de
% https://github.com/phist91/latex-templates

\documentclass[%
	paper=a4,
	fontsize=10pt,
	ngerman
	]{scrartcl}

% Basics für Codierung und Sprache
% ===========================================================
	\usepackage{scrtime}
	\usepackage{etex}
	\usepackage{shellesc}					% Compiler-Option -shell-escape benutzen!
	\usepackage[final]{graphicx}			% Einbindung von Grafiken
	\usepackage[utf8]{inputenc}				% Dateien sind UTF8-codiert
	\usepackage{babel}						% deutsche Silbentrennung, etc.
	\usepackage[german=quotes]{csquotes}	% deutsche Anführungszeichen mit \enquote{...}
% ===========================================================

% Fonts und Typographie
% ===========================================================
	\usepackage{sourcecodepro}
	\usepackage[default]{sourcesanspro}
	\usepackage{nimbusmononarrow}
	
	\usepackage[babel=true,final,tracking=smallcaps]{microtype}
	\DisableLigatures{encoding = T1, family = tt* } % keine Ligaturen für Monospace-Fonts
	\usepackage{ellipsis}
% ===========================================================

% Farben
% ===========================================================
	\usepackage[usenames,x11names,final]{xcolor}
% ===========================================================

% Mathe-Pakete und -Einstellungen
% ===========================================================
	\usepackage{mathtools}					% Tools zum Setzen von Formeln
	\usepackage{amssymb}					% übliche Mathe-Symbole
	\usepackage[bigdelims]{newtxmath}		% moderne Mathe-Font
	\allowdisplaybreaks						% seitenübergreifende Rechnungen
	\usepackage{bm}							% math bold font
	\usepackage{wasysym}					% noch mehr Symbole
% ===========================================================

% TikZ
% ===========================================================
	\usepackage{tikz}
	\usepackage{tikz-cd}					% kommutative Diagramme
	\usetikzlibrary{arrows.meta}			% mehr Pfeile!
	\usetikzlibrary{calc}					% TikZ kann rechnen
	\tikzset{>=Latex}						% Standard-Pfeilspitze
% ===========================================================

% Seitenlayout, Kopf-/Fußzeile
% ===========================================================
	\usepackage{scrpage2}
	\pagestyle{scrheadings}
	\usepackage[top=3cm, bottom=3cm, left=2.5cm, right=2cm]{geometry}
	\clearscrheadfoot 
	\setheadsepline{0.4pt}			 					% Linie in Kopfzeile
	\setfootsepline{0.4pt}								% Linie in Fußzeile
	\setkomafont{pagehead}{\bfseries}					% Schriftart Kopfzeile
	\setkomafont{pagefoot}{\normalfont\footnotesize}	% Schriftart Fußzeile 
	\cfoot{\thepage}									% Seitenzahl unten Mitte
	\lohead{\obenlinks}	% Titel oben links
	\raggedbottom							% Flattersatz
	\usepackage{setspace}					% erweiterte Abstandsoptionen
	\onehalfspacing							% Zeilenabstand 1.5-fach
	\setlength{\parindent}{0pt}				% Einrückung neuer Absätze
	\setlength{\parskip}{0.5\baselineskip}	% Abstand neuer Absätze
% ===========================================================

% Hyperref für Referenzen und Hyperlinks
% ===========================================================
	\usepackage[%
		hidelinks,
		pdfpagelabels,
		bookmarksopen=true,
		bookmarksnumbered=true,
		linkcolor=black,
		urlcolor=SkyBlue2,
		plainpages=false,
		pagebackref,
		citecolor=black,
		hypertexnames=true,
		pdfborderstyle={/S/U},
		linkbordercolor=SkyBlue2,
		colorlinks=false,
		backref=false]{hyperref}
	\hypersetup{final}
% ===========================================================

% Listen und Tabellen
% ===========================================================
	\usepackage{multicol}
	\usepackage[shortlabels]{enumitem}
	\setlist{itemsep=0pt}
	\setlist[enumerate]{font=\sffamily\bfseries}
	\setlist[itemize]{label=$\triangleright$}
	\usepackage{tabularx}
% ===========================================================

% minted
% ===========================================================
\usepackage{minted}
\setminted{%
	style=default,
	fontsize=\small,
	breaklines,
	breakanywhere=false,
	breakbytoken=false,
	breakbytokenanywhere=false,
	breakafter={.,},
	autogobble,
	numbersep=3mm,
	tabsize=4,
	linenos,
	frame=lines
}
\setmintedinline{%
	fontsize=\normalsize,
	numbers=none,
	numbersep=12pt,
	tabsize=4,
}

%%%%%%%%%%%%%%%%%%%%%%%%%%%%%%%%%%%%%%%%%%%%%%%%%%%%%%%%%%%
%%% Ab hier folgen nur noch vordefinierte Mathe-Befehle %%%
%%%%%%%%%%%%%%%%%%%%%%%%%%%%%%%%%%%%%%%%%%%%%%%%%%%%%%%%%%%

\newcommand{\BB}{\mathbb{B}}
\newcommand{\CC}{\mathbb{C}}
\newcommand{\NN}{\mathbb{N}}
\newcommand{\QQ}{\mathbb{Q}}
\newcommand{\RR}{\mathbb{R}}
\newcommand{\ZZ}{\mathbb{Z}}
\newcommand{\oh}{\mathcal{O}}						
\newcommand{\ol}[1]{\overline{#1}}
\newcommand{\wt}[1]{\widetilde{#1}}
\newcommand{\wh}[1]{\widehat{#1}}

\DeclareMathOperator{\id}{id} 				% Identität
\DeclareMathOperator{\pot}{\mathcal{P}}		% Potenzmenge

% Klammerungen und ähnliches
\DeclarePairedDelimiter{\absolut}{\lvert}{\rvert}		% Betrag
\DeclarePairedDelimiter{\ceiling}{\lceil}{\rceil}		% aufrunden
\DeclarePairedDelimiter{\Floor}{\lfloor}{\rfloor}		% aufrunden
\DeclarePairedDelimiter{\Norm}{\lVert}{\rVert}			% Norm
\DeclarePairedDelimiter{\sprod}{\langle}{\rangle}		% spitze Klammern
\DeclarePairedDelimiter{\enbrace}{(}{)}					% runde Klammern
\DeclarePairedDelimiter{\benbrace}{\lbrack}{\rbrack}	% eckige Klammern
\DeclarePairedDelimiter{\penbrace}{\{}{\}}				% geschweifte Klammern
\newcommand{\Underbrace}[2]{{\underbrace{#1}_{#2}}} 	% bessere Unterklammerungen
% Kurzschreibweisen für Faule und Code-Vervollständigung
\newcommand{\abs}[1]{\absolut*{#1}}
\newcommand{\ceil}[1]{\ceiling*{#1}}
\newcommand{\flo}[1]{\Floor*{#1}}
\newcommand{\no}[1]{\Norm*{#1}}
\newcommand{\sk}[1]{\sprod*{#1}}
\newcommand{\enb}[1]{\enbrace*{#1}}
\newcommand{\penb}[1]{\penbrace*{#1}}
\newcommand{\benb}[1]{\benbrace*{#1}}
\newcommand{\stack}[2]{\makebox[1cm][c]{$\stackrel{#1}{#2}$}}	% Präambel (ohne die geht nichts!)
\begin{document}
\begin{center}
	\begin{tabular}{|rlp{4cm}rl|}
	\hline
	 \textbf{Übungsblatt:} & 13 &  & \textbf{1. Abgabepartner*in:} & Alan Turing  \\
	     \textbf{Aufgabe:} & 37 &  & \textbf{2. Abgabepartner*in:} & Donald Knuth \\
	\textbf{Abgabegruppe:} & 42 &  & \textbf{3. Abgabepartner*in:} & Ada Lovelace \\ \hline
\end{tabular}
\end{center} 

Dies ist eine \LaTeX-Vorlage für Übungszettelabgaben im Fach Informatik, die das \texttt{minted}-Paket nutzt.
Im Gegensatz zu den standardmäßigen \texttt{listings}-Umgebungen, für die in meinem Git-Repository\footnote{\url{https://github.com/phist91/latex-templates}} ebenfalls eine Vorlage bereitgestellt wird, muss man einige Vorbereitungen treffen, damit \texttt{minted} (und auch diese Vorlage) benutzt werden kann.
Belohnt wird man dafür mit einem schöneren und deutlich differenzierteren automatischen Syntax Highlighting.
Nachfolgend gibt es ein paar Hinweise, wie diese Vorlage zu verwenden ist.

\subsection*{Disclaimer}
Ich kann leider grundsätzlich keinen \LaTeX-Support anbieten und verweise daher auf gängige Suchmaschinen und die \TeX-Community von StackExchange\footnote{\url{https://tex.stackexchange.org/}}.		
Sucht man den LaTeX-Befehl für ein bestimmtes Symbol, ist Detexify\footnote{\url{http://detexify.kirelabs.org/}} praktisch.

\subsection*{Verwendung}
\texttt{minted} basiert auf der Programmiersprache Python\footnote{\url{https://www.python.org/downloads/}}.
Damit man es zum Laufen bekommt, muss man allerdings keine Kenntnisse diesbezüglich mitbringen, sondern lediglich eine aktuelle Version installieren.
Im Anschluss muss noch \texttt{Pygments} eingerichtet werden.
Details dazu finden sich in Abschnitt 2.1 der \texttt{minted}-Dokumentation\footnote{\url{https://ctan.org/pkg/minted?lang=de}}.

Als Test sollte man versuchen, die Datei \texttt{abgaben-minted.tex} in einem TeX-Editor zu öffnen und zu kompilieren.
Dabei handelt es sich um den TeX-Code zu genau diesem Dokument.
Wenn das nicht klappt, kann das folgende Gründe haben:
\begin{itemize}
	\item In den Umgebungsvariablen des Systems fehlt das Verzeichnis der \LaTeX-Distribution oder zu Python.
	\item Der Compiler wird nicht mit der Shell-Escape-Option aufgerufen.
		Ggfs. sollte in den Einstellungen des Editors nachgeschaut werden, dass \texttt{pdflatex} mit der Option \texttt{-shell-escape} aufgerufen wird.
	\item Irgendwas anderes. Man prüfe die Ausgabe des Compilers und ziehe ggfs. die \texttt{minted}-Dokumentation zurate.
\end{itemize}
Wenn alles geklappt hat, kann man diese Vorlage verwenden.
Ich habe als Einstiegshilfe unten drei Beispiele zur Einbindung von C-Code eingefügt.
Die Verwendung der entsprechenden TeX-Befehle sollte anhand der Datei \texttt{abgaben-minted.tex} erschließbar sein.
Weitere Hinweise und Konfigurationsmöglichkeiten findet man in der Paketdokumentation.

\textbf{Nicht vergessen:} Ein automatisches Syntax Highlighting entbindet \textbf{niemals} von der Pflicht, gut lesbaren und gut nachvollziehbaren Code zu produzieren.

\begin{flushright}
Phil Steinhorst \\
\url{https://github.com/phist91/latex-templates}
\end{flushright}

\newpage

%%%%%%%%%%%%%%%%%%%%%%%%%%%%%%%%%%%%%%%%%%%%%%%%%%%%%%%%%%%%%%%

\subsection*{Beispiel für direkt eingegebenen C-Code:}
\begin{minted}{c}
#include <stdio.h>

int main(int argc, char** argv){
	int i;
	for(i = 0; i < argc; i++){
		printf("%s \n", argv[i]);
	}
	return 0;
}
\end{minted}

%%%%%%%%%%%%%%%%%%%%%%%%%%%%%%%%%%%%%%%%%%%%%%%%%%%%%%%%%%%%%%%
\subsection*{Beispiel für C-Code, der aus einer eigenen Datei eingebunden wird:}

\inputminted{c}{programm.c}


%%%%%%%%%%%%%%%%%%%%%%%%%%%%%%%%%%%%%%%%%%%%%%%%%%%%%%%%%%%%%%%

\textbf{Beispiel für Inline-C-Code:} \mintinline{c}{int main(int argc, char** argv)}

%%%%%%%%%%%%%%%%%%%%%%%%%%%%%%%%%%%%%%%%%%%%%%%%%%%%%%%%%%%%%%%


\end{document}